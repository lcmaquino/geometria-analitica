\documentclass[12pt,a4paper]{article}
\usepackage[utf8]{inputenc}
\usepackage[brazil]{babel}
\usepackage{graphicx}
\usepackage{amssymb, amsfonts, amsmath}
\usepackage{float}
\usepackage{enumerate}
\usepackage[top=2.5cm, bottom=2.5cm, left=1.25cm, right=1.25cm]{geometry}

\begin{document}
\pagestyle{empty}

\begin{center}
  \begin{tabular}{ccc}
    \begin{tabular}{c}
      \includegraphics[scale=0.25]{../../../biblioteca/imagem/brasao-de-armas-brasil} \\
    \end{tabular} & 
    \begin{tabular}{c}
      Ministério da Educação \\
      Universidade Federal dos Vales do Jequitinhonha e Mucuri \\
      Faculdade de Ciências Sociais, Aplicadas e Exatas - FACSAE \\
      Departamento de Ciências Exatas - DCEX \\
      Disciplina: Geometria Analítica \quad Semestre: 2020/5\\
      Prof. Me. Luiz C. M. de Aquino\\
    \end{tabular} &
    \begin{tabular}{c}
      \includegraphics[scale=0.25]{../../../biblioteca/imagem/logo-ufvjm} \\
    \end{tabular}
  \end{tabular}
\end{center}

\begin{center}
  \textbf{Lista IV}
\end{center}

\begin{enumerate}

 \item Determine a equação simétrica da reta que passa pelo ponto
 $P = (3,\, 4,\, -2)$ e é paralela à reta:
 $$
   s:\begin{cases}
       x = -1 + 2t\\
       y = 2 - t\\
       z = 1 + t\\
     \end{cases}
 $$
 
 \item Determine o ponto de interseção entre as retas:
 $$
   r: \dfrac{x + 1}{-1} = \dfrac{y - 1}{3} = \dfrac{z - 2}{2}
   \quad \textrm{ e }\quad
   s:\begin{cases}
     x = 8 + 2t\\
     y = -11 - t\\
     z = 2 + 2t\\
   \end{cases}
 $$ 

 \item Seja $P$ o ponto de interseção entre as retas:
 $$
   r:\begin{cases}
     x = 3 + t\\
     y = -2 - t\\
     z = 3 + t\\
   \end{cases}
   \quad \textrm{ e }\quad
   s:\begin{cases}
     x = 3 + m\\
     y = 2 + m\\
     z = -1 - m\\
   \end{cases}
 $$ 
 
 Determine a reta $q$ perpendicular ao mesmo tempo às retas $r$ e $s$ e que
 passa pelo ponto $Q = (1,\,1,\,-1)$

 \item Determine a reta que é a interseção entre os planos $x - y + 2z - 1 = 0$ 
 e $x - y - z - 5 = 0$.
  
 \item Ache a equação do plano paralelo ao plano $2x - y + 5z - 3 = 0$ e que 
 passa por $P = (1,\, -2,\, 1 )$.

 \item Dadas as retas
 $$
   r: \dfrac{x-2}{2} = \dfrac{y}{2} = z \quad \textrm{ e }\quad s: x - 2 = y = z,
 $$

 obtenha uma equação geral para o plano determinado por $r$ e $s$.

 \item Seja o plano $\pi$ que passa pela origem e é perpendicular à reta que une os pontos
  $A = (1,\, 0,\, 0)$ e $B = (0,\, 1,\, 0)$. Determine a equação geral de $\pi$.
  
 \item Seja $ax + by + cz + d = 0$ a equação do plano $\pi$ com $abcd \neq 0$.
 \begin{enumerate}
  \item Determine a interseção de $\pi$ com os eixos;
  \item Se $P_1 = (p_1,\,0,\,0)$, $P_2 = (0,\,p_2,\,0)$ e $P_3 = (0,\,0,\,p_3)$ são as interseções de $\pi$ com
  os eixos, prove que a equação de $\pi$ pode ser escrita como:
  $$\frac{x}{p_1} + \frac{y}{p_2} + \frac{z}{p_3} = 1$$
 \end{enumerate}
 
\end{enumerate}

\end{document}
