\documentclass[12pt,a4paper]{article}
\usepackage[utf8]{inputenc}
\usepackage[brazil]{babel}
\usepackage{graphicx}
\usepackage{amssymb, amsfonts, amsmath}
\usepackage{float}
\usepackage{enumerate}
\usepackage[top=2.5cm, bottom=2.5cm, left=1.25cm, right=1.25cm]{geometry}

\begin{document}
\pagestyle{empty}

\begin{center}
  \begin{tabular}{ccc}
    \begin{tabular}{c}
      \includegraphics[scale=0.25]{../../../biblioteca/imagem/brasao-de-armas-brasil} \\
    \end{tabular} & 
    \begin{tabular}{c}
      Ministério da Educação \\
      Universidade Federal dos Vales do Jequitinhonha e Mucuri \\
      Faculdade de Ciências Sociais, Aplicadas e Exatas - FACSAE \\
      Departamento de Ciências Exatas - DCEX \\
      Disciplina: Geometria Analítica \quad Semestre: 2020/5\\
      Prof. Me. Luiz C. M. de Aquino\\
    \end{tabular} &
    \begin{tabular}{c}
      \includegraphics[scale=0.25]{../../../biblioteca/imagem/logo-ufvjm} \\
    \end{tabular}
  \end{tabular}
  Aluno(a): \rule{0.5\textwidth}{0.01cm} \quad Data: \rule{0.6cm}{0.01cm} / \rule{0.6cm}{0.01cm} / \rule{1.25cm}{0.01cm}
\end{center}

\begin{center}
 \textbf{Exame Final}
\end{center}

\textbf{Instruções}
\begin{itemize}
 \item Todas as justificativas necessárias na solução de cada questão devem
   estar presentes nesta avaliação;
 \item As respostas finais de cada questão devem estar escritas de caneta;
 \item Esta avaliação tem um total de 100,0 pontos.
\end{itemize}

\begin{enumerate}

  \item \textbf{[20,0 pontos]} Dados os pontos $A = (1,\, -2,\, -3)$, 
    $B = (-5,\, 2,\, -1)$ e $C = (4,\, 0,\, -1)$, determine o ponto $D$ tal que
    $ABCD$ seja um paralelogramo.
  
  \item \textbf{[20,0 pontos]} Prove que as diagonais de um paralelogramo se
    cruzam ao meio. (Sugestão: considerando que $M$ e $N$ são os pontos médios
    das diagonais do paralelogramo, prove que $\overrightarrow{MN} = \vec{0}$
    e conclua que $M = N$.)

  \item \textbf{[20,0 pontos]} Verifique se o conjunto 
    $B = \{v_1 = (-1,\, 2,\, 3),\, v_2 = (1,\, -3,\, -1),\, v_3 = (6,\, -16,\, 10)\}$
    é LI ou LD.
    
  \item \textbf{[20,0 pontos]} A reta $r$ passa pelo ponto $P = (-1,\, 2,\, 4)$
    e pela interseção entre o plano $\pi : x + y - z + 1 = 0$ e a reta 
    $s : \dfrac{x - 5}{5} = \dfrac{y - 11}{3} = \dfrac{z - 9}{2}$. Determine 
    as equações paramétricas da reta $r$.

  \item \textbf{[20,0 pontos]} O plano $\pi$ passa pelo ponto
    $P = (-1,\, 4,\, 5)$ e contém a reta
    $r : \begin{cases} x = 1 - t \\ y = 3 + t \\ z = -1 + 2t\end{cases}$.
    Determine a equação geral do plano $\pi$.

\end{enumerate}

\end{document}
