\documentclass[12pt,a4paper]{article}
\usepackage[utf8]{inputenc}
\usepackage[brazil]{babel}
\usepackage{graphicx}
\usepackage{amssymb, amsfonts, amsmath}
\usepackage{float}
\usepackage{enumerate}
\usepackage[top=2.5cm, bottom=2.5cm, left=1.25cm, right=1.25cm]{geometry}

\begin{document}
\pagestyle{empty}

\begin{center}
  \begin{tabular}{ccc}
    \begin{tabular}{c}
      \includegraphics[scale=0.25]{../../../biblioteca/imagem/brasao-de-armas-brasil} \\
    \end{tabular} & 
    \begin{tabular}{c}
      Ministério da Educação \\
      Universidade Federal dos Vales do Jequitinhonha e Mucuri \\
      Faculdade de Ciências Sociais, Aplicadas e Exatas - FACSAE \\
      Departamento de Ciências Exatas - DCEX \\
      Disciplina: Geometria Analítica \quad Semestre: 2020/5\\
      Prof. Me. Luiz C. M. de Aquino\\
    \end{tabular} &
    \begin{tabular}{c}
      \includegraphics[scale=0.25]{../../../biblioteca/imagem/logo-ufvjm} \\
    \end{tabular}
  \end{tabular}
\end{center}

\begin{center}
  \textbf{Lista II}
\end{center}

\begin{enumerate}

  \item Dados os pontos $A = (1;\, -2;\, -3)$, $B = (-5;\, 2;\, -1)$ e $C = (4;\, 0;\, -1)$, 
  determine o ponto $D$ tal que $ABCD$ seja de um paralelogramo.
   
  \item Determine a área de um paralelogramo que tem três vértices consecutivos dados 
  pelos pontos $\left(\sqrt{2};\, -1;\, 1\right)$ , $\left(1;\, \dfrac{1}{2};\, -1\right)$ e 
  $\left(1;\, 1;\, -2\right)$.

  \item Ache o vetor unitário da bissetriz do ângulo entre os vetores 
  $\vec{v} = 2\vec{i} + 2\vec{j} + \vec{k}$ e $\vec{w} = 6\vec{i} + 2\vec{j} - 3\vec{k}$.
  (Sugestão: observe que a soma de dois vetores está na direção da bissetriz se, e somente se,
  os dois tiverem o mesmo comprimento. Portanto, tome múltiplos escalares de $\vec{v}$ e
  $\vec{w}$ de forma que eles tenham o mesmo comprimento e tome o vetor unitário na
  direção da soma deles.)
    
  \item Prove que as diagonais de um quadrado são ortogonais.
    
  \item Seja o triângulo $ABC$ com $M$ e $N$ os pontos médios dos lados $\overline{AB}$ e $\overline{AC}$, respectivamente. Prove 
  que $\overrightarrow{MN} = \dfrac{1}{2}\overrightarrow{BC}$.
  
\end{enumerate}

\end{document}
