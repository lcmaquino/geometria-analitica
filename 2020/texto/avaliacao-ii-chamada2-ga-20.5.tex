\documentclass[12pt,a4paper]{article}
\usepackage[utf8]{inputenc}
\usepackage[brazil]{babel}
\usepackage{graphicx}
\usepackage{amssymb, amsfonts, amsmath}
\usepackage{float}
\usepackage{enumerate}
\usepackage[top=2.5cm, bottom=2.5cm, left=1.25cm, right=1.25cm]{geometry}

\begin{document}
\pagestyle{empty}

\begin{center}
  \begin{tabular}{ccc}
    \begin{tabular}{c}
      \includegraphics[scale=0.25]{../../../biblioteca/imagem/brasao-de-armas-brasil} \\
    \end{tabular} & 
    \begin{tabular}{c}
      Ministério da Educação \\
      Universidade Federal dos Vales do Jequitinhonha e Mucuri \\
      Faculdade de Ciências Sociais, Aplicadas e Exatas - FACSAE \\
      Departamento de Ciências Exatas - DCEX \\
      Disciplina: Geometria Analítica \quad Semestre: 2020/5\\
      Prof. Me. Luiz C. M. de Aquino\\
    \end{tabular} &
    \begin{tabular}{c}
      \includegraphics[scale=0.25]{../../../biblioteca/imagem/logo-ufvjm} \\
    \end{tabular}
  \end{tabular}
  Aluno(a): \rule{0.5\textwidth}{0.01cm} \quad Data: \rule{0.6cm}{0.01cm} / \rule{0.6cm}{0.01cm} / \rule{1.25cm}{0.01cm}
\end{center}

\begin{center}
 \textbf{Avaliação II - 2ª chamada.}
\end{center}

\textbf{Instruções}
\begin{itemize}
 \item Todas as justificativas necessárias na solução de cada questão devem
   estar presentes nesta avaliação;
 \item As respostas finais de cada questão devem estar escritas de caneta;
 \item Esta avaliação tem um total de 25,0 pontos.
\end{itemize}

\begin{enumerate}

  \item \textbf{[5,0 pontos]} Em cada item abaixo, verifique se é possível
  escrever o vetor $u = (-1,\, 2,\, 3)$ como combinação linear dos vetores 
  $v$ e $w$:
    \begin{enumerate}[(a)]
      \item $v = (2,\, 1,\, 6)$,  $w = (3,\, -1,\, 3)$
      \item $v = (-5,\, 8,\, 1)$, $w = (4,\, -6,\, 1)$
    \end{enumerate}
  
  \item \textbf{[5,0 pontos]} Verifique se cada conjunto abaixo é LI ou LD.
    \begin{enumerate}[(a)]
      \item $B = \{v_1 = (-1,\, 2,\, 3),\, v_2 = (1,\, -3,\, -1),\, v_3 = (6,\, -16,\, 10)\}$
      \item $C = \{v_1 = (1,\, -1,\, 0,\, 2), v_2 = (3,\, 1,\, -1,\, 4),\, v_3 = (-7,\, -5,\, 3,\, -8)\}$
    \end{enumerate}

  \item \textbf{[5,0 pontos]} Seja o conjunto $B = \{v_1 = (-1,\, 2),\, v_2 = (3,\, 5)\}$.
    \begin{enumerate}[(a)]
      \item Verifique que $B$ é LI.
      \item Verifique que $B\cup\{v_3 = (-10,\ -2)\}$ é LD.
      \item Determine $v_3 = (-10,\, -2)$ como combinação linear dos elementos de $B$.
    \end{enumerate}
    
  \item \textbf{[5,0 pontos]} Prove que se $\{v_1,\, v_2,\, \ldots,\, v_n\}$ é LI e o vetor $v_{n+1}$ não
    pode ser escrito como combinação linear de $\{v_1,\, v_2,\, \ldots,\, v_n\}$,
    então $\{v_1,\, v_2,\, \ldots,\, v_n,\, v_{n+1}\}$ é LI.
   
  \item \textbf{[5,0 pontos]} Prove que se $\{v_1,\,v_2,\,\ldots,\,v_n\}$ é LI
    e $\{v_1,\,v_2,\,\ldots,\,v_n,\,v_{n+1}\}$ é LD, então $v_{n+1}$ é
    combinação linear de $v_1,\,v_2,\,\ldots,\,v_n$.
    
 
\end{enumerate}

\end{document}
