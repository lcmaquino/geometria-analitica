\documentclass[12pt,a4paper]{article}
\usepackage[utf8]{inputenc}
\usepackage[brazil]{babel}
\usepackage{graphicx}
\usepackage{amssymb, amsfonts, amsmath}
\usepackage{float}
\usepackage{enumerate}
\usepackage[top=2.5cm, bottom=2.5cm, left=1.25cm, right=1.25cm]{geometry}

\begin{document}
\pagestyle{empty}

\begin{center}
  \begin{tabular}{ccc}
    \begin{tabular}{c}
      \includegraphics[scale=0.25]{../../../biblioteca/imagem/brasao-de-armas-brasil} \\
    \end{tabular} & 
    \begin{tabular}{c}
      Ministério da Educação \\
      Universidade Federal dos Vales do Jequitinhonha e Mucuri \\
      Faculdade de Ciências Sociais, Aplicadas e Exatas - FACSAE \\
      Departamento de Ciências Exatas - DCEX \\
      Disciplina: Geometria Analítica \quad Semestre: 2020/5\\
      Prof. Me. Luiz C. M. de Aquino\\
    \end{tabular} &
    \begin{tabular}{c}
      \includegraphics[scale=0.25]{../../../biblioteca/imagem/logo-ufvjm} \\
    \end{tabular}
  \end{tabular}
  Aluno(a): \rule{0.5\textwidth}{0.01cm} \quad Data: \rule{0.6cm}{0.01cm} / \rule{0.6cm}{0.01cm} / \rule{1.25cm}{0.01cm}
\end{center}

\begin{center}
 \textbf{Avaliação III - 2ª chamada.}
\end{center}

\textbf{Instruções}
\begin{itemize}
 \item Todas as justificativas necessárias na solução de cada questão devem
   estar presentes nesta avaliação;
 \item As respostas finais de cada questão devem estar escritas de caneta;
 \item Esta avaliação tem um total de 25,0 pontos.
\end{itemize}

\begin{enumerate}

  \item \textbf{[5,0 pontos]} Determine o ponto de interseção entre as retas:
 $$
   r: \dfrac{x + 1}{-1} = \dfrac{y - 1}{3} = \dfrac{z - 2}{2}
   \quad \textrm{ e }\quad
   s:\begin{cases}
     x = 8 + 2t\\
     y = -11 - t\\
     z = 2 + 2t\\
   \end{cases}
 $$ 
  
  \item \textbf{[5,0 pontos]} Sejam as retas:
 $$
   r:\begin{cases}
     x = 3 + t\\
     y = -2 - t\\
     z = 3 + t\\
   \end{cases}
   \quad \textrm{ e }\quad
   s:\begin{cases}
     x = 3 + m\\
     y = 2 + m\\
     z = -1 - m\\
   \end{cases}
 $$ 
 
 Determine a reta $q$ perpendicular ao mesmo tempo às retas $r$ e $s$ e que
 passa pelo ponto $Q = (1,\,1,\,-1)$

  \item \textbf{[5,0 pontos]} Dadas as retas
 $$
   r: \dfrac{x-2}{2} = \dfrac{y}{2} = z \quad \textrm{ e }\quad s: x - 2 = y = z,
 $$

 obtenha uma equação geral para o plano determinado por $r$ e $s$.
    
  \item \textbf{[5,0 pontos]} A reta $r$ passa pelo ponto $P = (-1,\, 2,\, 4)$
  e pela interseção entre o plano $\pi : x + y - z + 1 = 0$ e a reta 
  $s : \dfrac{x - 5}{5} = \dfrac{y - 11}{3} = \dfrac{z - 9}{2}$. Determine 
  a equação da reta $r$.

  \item \textbf{[5,0 pontos]} O plano $\pi$ passa pelo ponto $P = (-1,\, 4,\, 5)$ e 
  contém a reta $r : \begin{cases} x = 1 - t \\ y = 3 + t \\ z = -1 + 2t\end{cases}$. 
  Determine a equação geral do plano $\pi$.
    
 
\end{enumerate}

\end{document}
