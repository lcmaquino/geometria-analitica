\documentclass[12pt,a4paper]{article}
\usepackage[utf8]{inputenc}
\usepackage[brazil]{babel}
\usepackage{graphicx}
\usepackage{amssymb, amsfonts, amsmath}
\usepackage{float}
\usepackage{enumerate}
\usepackage[top=2.5cm, bottom=2.5cm, left=1.25cm, right=1.25cm]{geometry}

\begin{document}
\pagestyle{empty}

\begin{center}
  \begin{tabular}{ccc}
    \begin{tabular}{c}
      \includegraphics[scale=0.25]{../../../biblioteca/imagem/brasao-de-armas-brasil} \\
    \end{tabular} & 
    \begin{tabular}{c}
      Ministério da Educação \\
      Universidade Federal dos Vales do Jequitinhonha e Mucuri \\
      Faculdade de Ciências Sociais, Aplicadas e Exatas - FACSAE \\
      Departamento de Ciências Exatas - DCEX \\
      Disciplina: Geometria Analítica \quad Semestre: 2020/5\\
      Prof. Me. Luiz C. M. de Aquino\\
    \end{tabular} &
    \begin{tabular}{c}
      \includegraphics[scale=0.25]{../../../biblioteca/imagem/logo-ufvjm} \\
    \end{tabular}
  \end{tabular}
\end{center}

\begin{center}
  \textbf{Lista III}
\end{center}

\begin{enumerate}

  \item Em cada item abaixo, verifique se é possível escrever o vetor 
    $u = (-1,\, 2,\, 3)$ como combinação linear dos vetores $v$ e $w$:
    \begin{enumerate}[(a)]
      \item $v = (2,\, 1,\, 6)$,  $w = (3,\, -1,\, 3)$
      \item $v = (-5,\, 8,\, 1)$, $w = (4,\, -6,\, 1)$
      \item $v = (1,\, 0,\, -2)$, $w = \left(2,\, -1,\,-\dfrac{3}{2}\right)$
      \item $v = \left(1,\, \dfrac{3}{4},\, -1\right)$, $w = \left(1,\, -\dfrac{1}{6},\, -\dfrac{5}{3}\right)$
    \end{enumerate}
   
  \item Verifique se o conjunto 
    $B = \{v_1 = (-1,\, 2,\, 3),\, v_2 = (1,\, -3,\, -1),\, v_3 = (6,\, -16,\, 10)\}$ é
    LI ou LD.

  \item Seja $P$ o espaço vetorial formado por todas as funções polinomiais de
    grau menor ou igual a 2. Considerando esse espaço vetorial, verifique se o conjunto
    abaixo é LI ou LD: 
    $$\left\{p_1(x) = -x^2 + \dfrac{x}{2} - \dfrac{5}{2},\, p_2(x) = x^2 + 1,\, p_3(x) = 6x^2  - x + 9\right\}$$
    
  \item Prove que $\{v_1,\,v_2\}$ é LD se e somente se $v_1 =\alpha v_2$ ou $v_2 =\alpha v_1$.
    
  \item Prove que se $\{v_1,\, v_2,\, \ldots,\, v_n\}$ é LI e o vetor $v_{n+1}$ não
    pode ser escrito como combinação linear de $\{v_1,\, v_2,\, \ldots,\, v_n\}$,
    então $\{v_1,\, v_2,\, \ldots,\, v_n,\, v_{n+1}\}$ é LI.
  
\end{enumerate}

\end{document}
