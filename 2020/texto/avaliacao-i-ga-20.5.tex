\documentclass[12pt,a4paper]{article}
\usepackage[utf8]{inputenc}
\usepackage[brazil]{babel}
\usepackage{graphicx}
\usepackage{amssymb, amsfonts, amsmath}
\usepackage{float}
\usepackage{enumerate}
\usepackage[top=2.5cm, bottom=2.5cm, left=1.25cm, right=1.25cm]{geometry}

\begin{document}
\pagestyle{empty}

\begin{center}
  \begin{tabular}{ccc}
    \begin{tabular}{c}
      \includegraphics[scale=0.25]{../../../biblioteca/imagem/brasao-de-armas-brasil} \\
    \end{tabular} & 
    \begin{tabular}{c}
      Ministério da Educação \\
      Universidade Federal dos Vales do Jequitinhonha e Mucuri \\
      Faculdade de Ciências Sociais, Aplicadas e Exatas - FACSAE \\
      Departamento de Ciências Exatas - DCEX \\
      Disciplina: Geometria Analítica \quad Semestre: 2020/5\\
      Prof. Me. Luiz C. M. de Aquino\\
    \end{tabular} &
    \begin{tabular}{c}
      \includegraphics[scale=0.25]{../../../biblioteca/imagem/logo-ufvjm} \\
    \end{tabular}
  \end{tabular}
  Aluno(a): \rule{0.5\textwidth}{0.01cm} \quad Data: \rule{0.6cm}{0.01cm} / \rule{0.6cm}{0.01cm} / \rule{1.25cm}{0.01cm}
\end{center}

\begin{center}
 \textbf{Avaliação I}
\end{center}

\textbf{Instruções}
\begin{itemize}
 \item Todas as justificativas necessárias na solução de cada questão devem estar presentes nesta avaliação;
 \item As respostas finais de cada questão devem estar escritas de caneta;
 \item Esta avaliação tem um total de 25,0 pontos.
\end{itemize}

\begin{enumerate}

  \item \textbf{[5,0 pontos]} Represente geometricamente dois vetores $\vec{u}$ e $\vec{v}$ que possuem
  \textbf{apenas}:
  
    \begin{enumerate}[(a)]
      \item a mesma direção;
      \item o mesmo sentido e mesma direção;
      \item a mesma magnitude (ou comprimento) e mesma direção;
    \end{enumerate}
  
  \item \textbf{[5,0 pontos]} Classifique as afirmações em Verdadeiro ou Falso.
  
  \begin{description}
    \item[(\quad)] O vetor $-2\vec{u}$ tem o mesmo sentido de $\vec{u}$, mas tem 
      direção contrária.
    \item[(\quad)] O vetor $-2\vec{u}$ tem a metade do comprimento de $\vec{u}$.
    \item[(\quad)] Se $\vec{u}$ e $\vec{v}$ possuem a mesma direção, sentido e
      comprimento, então $\vec{u} = \vec{v}$.
    \item[(\quad)] Para qualquer vetor $\vec{u}$, temos que
      $\vec{u} + (-\vec{u}) = \vec{0}$
    \item[(\quad)] O comprimento do vetor $\lambda u$ é diferente do comprimento 
      do vetor $-\lambda u$.
    \item[(\quad)] Sendo $A$, $B$, $C$ e $D$ pontos quaisquer, temos que 
      $\overrightarrow{AB} - \overrightarrow{CB} + \overrightarrow{CD} = \overrightarrow{AD}$.
  \end{description}
  
  \item \textbf{[5,0 pontos]} Dados os pontos $A = (1;\, -2;\, -3)$, $B = (-5;\, 2;\, -1)$ e $C = (4;\, 0;\, -1)$, 
  determine o ponto $D$ tal que $ABCD$ seja um paralelogramo.
   
  \item \textbf{[5,0 pontos]} Determine a área do triângulo que tem vértices dados 
  pelos pontos $\left(\sqrt{2};\, -1;\, 1\right)$ , $\left(1;\, \dfrac{1}{2};\, -1\right)$ e 
  $\left(1;\, 1;\, -2\right)$.
    
  \item \textbf{[5,0 pontos]} Prove que $\|\vec{u}\times\vec{v}\|$ é igual
  a área do paralelogramo com lados representados pelos vetores $\vec{u}$ e $\vec{v}$.
  
\end{enumerate}

\end{document}
